\section{Einführung}
\label{sec:einfuehrung}

Die Funktionsfähigkeit eines Bauteils hängt im Wesentlichen von seiner Maß- und Formhaltigkeit ab. In jedem Betrieb, welcher Werkstücke herstellt, ist das oberste Prinzip, Ausschuss zu vermeiden und somit einen möglichst hohen Ertrag zu erwirtschaften. Die Produktion von Ausschussteilen bedeutet einen verringerten Gewinn für das Unternehmen. Da bei jedem Fertigungsprozess Abweichungen von der idealen Gestalt des Bauteiles auftreten, ist die Aufgabe des Konstrukteurs, die Grenzen der Bearbeitungsungenauigkeiten und Fehler festzulegen, bei der die Anforderungen an das Werkstück noch zufriedenstellend, sowohl in Hinblick auf Funktionalität als auch Lebensdauer, erfüllt werden.\\
Die Ermittlung der Bauteilgüte fällt in das Gebiet der Fertigungsmesstechnik. Mithilfe dieser Disziplin des Ingenieurwesens sollen die Abweichungen des Bauteils von der Idealgestalt festgestellt und beurteilt werden. Der Messtechnik ist es allerdings nicht möglich, die gesamte, tatsächliche Oberfläche eines Bauteils zu vermessen. Es kann nur die Lage diskreter Punkte, mit einer bestimmt, dem Messverfahren eigenen Genauigkeit, bestimmt werden. Dies bildet aber nicht die kontinuierliche Formgestalt der technischen Oberfläche ab. Um eine optimale Messstrategie zu wählen wäre es von Vorteil, die am Bauteil zu erwartenden Abweichungen simulieren zu können. So könnte das Muster und die Anzahl der Messpunkte bereits vor der Fertigung des Bauteils angepasst. Kennt man den Einfluss der verschiedenen Prozessfaktoren, wie Maschinenparameter, Verschleißzustand des Werkzeugs oder Umgebungsbedingungen, auf die Güte des Bauteils, so lässt sich diese konkret vorhersagen. Dies würde eine rechtzeitige Anpassung der Verfahrensparameter erlauben, was zu einer Reduktion der Ausschussteile beiträgt. Erreicht die Genauigkeit solch einer Simulation der Fertigungsabweichungen einen Grad zuverlässiger Vorhersage, so könnt man den Messaufwand perspektivisch drastisch reduzieren, bzw. auf Messungen verzichten, da man mit Kenntnis und Überwachung der Produktionsparameter, vorhersagen kann, wie genau das Bauteil gefertigt wird.

Um diese Vision realisieren zu können, soll in dieser Arbeit geprüft werden, ob es möglich ist, anhand des verwendeten Fertigungsverfahrens Abweichungen von der Idealgestalt des Bauteils vorherzusagen und welche Prozessparameter, zu welchem Fehler führen. 
Weiterhin soll ein Prototyp einer Software entwickelt werden, die eine vom Nutzer bereitgestellte Idealgeometrie, nach seinen Anforderungen, diskretisieren soll. Das erstellte Programm wird diese Punkte anschließend, anhand verschiedener Funktionen, verändern. Dies soll eine Deformation der Geometrieelemente der Bauteiloberfläche simulieren. Die veränderten Punkte sollen vom Programm als Ergebnis ausgegeben werden. Die ausgegebenen Punktkoordinaten lassen Rückschluss auf die zu erwartende Formgestalt des Werkstückes zu und sind Ausgangspunkt zur Beurteilung dessen Güte.          




\subsection{Typische Abweichung verschiedener spanender Fertigungsvefahren}

\subsubsection{Ursachen für Fertigungsabweichungen}

Jedes Bauteil, welches in einem Fertigungsverfahren hergestellt wird entspricht in seiner Realgestalt nicht der in den Fertigungsunterlagen (Technische Zeichnung oder CAD-Model) festgehaltenen Idealgeometrie. Es sind durch verschiedene Einflüsse stets Abweichungen von der Geometrischen Oberfläche des Objektes festzustellen. Dies hat nach [Denkena] verschiedene Ursachen, die im Werkstück selbst, dem bearbeitenden Werkzeug, der Werkzeugmaschine oder Fertigungsumgebung liegen. 

So werden als Ursachen für Fertigungsabweichungen, die durch das Werkzeug bestimmt werden bereits vorhandene Form- und Lagetoleranzen des Rohteils, Festigkeitsunterschiede der abzuspanenden Teile, Auslösung bzw. Einbringung von Eigenspannungen in das Werkstück und örtlich und zeitlich veränderliche Temperaturfelder im Bauteil genannt.
Es ist festzuhalten, dass Wärmeeinbringung am Werkstück vor allem Formfehler zur Folge hat. Bei zylindrischen Bauteilen zeigen sich diese hauptsächlich in axialer Richtung, wegen inhomogener elastischer Erwärmung.

Weiterhin hat auch das bearbeitende Werkzeug einen sehr großen Einfluss auf die wirkliche Geometrie des Bauteils. Hier sind besonders die Nachgiebigkeit des Werkzeugs bzw. Werkzeughalters, die Lageabweichung des Werkzeugs beim Werkzeugwechsel sowie der Verschleiß des Werkzeugs zu nennen. Dabei hat der Werkzeugverschleiß einen maßgeblichen Einfluss auf die Maß- und Formgenauigkeit. Werkzeugverschleiß führt zu Schneidenversatz, dies wiederum begründet die Fehler, die die Maßhaltigkeit betreffen. Die Formabweichungen sind in den durch den Werkzeugverschleiß auftretenden höheren Schneidkräften bedingt. Diese führen zusammen mit der Nachgiebigkeit Werkzeugmaschine zum auftreten dieser Abweichungen.

Der Einfluss der bearbeitenden Maschine liegt in ihrer Nachgiebigkeit im Kraftfluss und ihrer thermischen Wirkung. So kommt es zu geometrischen Abweichungen und thermisch bedingten Verformungen während des Bearbeitungsprozesses. Ungenauigkeiten in den Führungen der Werkzeugmaschine führen über die kinematischen Kette der Maschine zu sich fortpflanzenden geometrischen Fehlern im Bauteil [Neugebauer] 

Auch die Umgebung der Maschine wirkt natürlich auf die Werkstückqualität ein. So führen externe Wärmequellen, Änderungen der Umgebungstemperatur oder eine Veränderung der Kühlschmierung zu Abweichungen am Werkstück. Zählt man den Bearbeiter des Bauteils zu den Umgebungsfaktoren hinzu, ergeben sich auch durch systematische oder zufällige menschliche Fehler in der Einrichtung und Überwachung des Bearbeitungsprozesses, Einflüsse auf die Formgestalt des Bauteils. 

Es ist festzuhalten, dass ein komplexes System an Einflussfaktoren auf das Werkstück einwirken, die sich gegenseitig beeinflussen können. Es ist nicht möglich, ohne detaillierte Kenntnis aller Faktoren bereits im Vorfeld konkret vorherzusagen, welche Abweichungen bei einem Fertigungsverfahren zu erwarten sind. Durch Erfahrungswerte lassen sich allerdings typische Abweichungen für die einzelnen Fertigungsverfahren beschreiben. 

\subsubsection {Fertigungsabweichungen beim Drehen}

Beim Fertigungsverfahren Drehen sind, je nach konkretem Drehverfahren, unterschiedliche Maßtoleranzen zu erreichen. So sind laut [Dietrich.2014] beim Schlichtdrehen Toleranzen von IT7 bis IT8 realistisch, beim Feinschlichten können unter optimalen Drehbedingungen sogar Genauigkeiten von IT6 umgesetzt werden. Das deckt sich mit den Angaben nach [Denkena.2011] wonach mit konventionellen Drehmaschinen Maßabweichungen von IT6 bis IT7 realisierbar sind. Ergänzend dazu werden mögliche Rundheitsabweichungen auf konventionellen Drehmaschinen von von weniger als 2,5µm, Zylindrizitätsabweichungen von weniger als 2µm und gemittelte Rautiefen $R_{z}$ von 2 bis 6µm beschrieben. 

Außerdem wird die mögliche Fertigungsgenauigkeit von Präzisionsdrehmaschinen genannt. Dabei werden Maßabweichungen im Bereich von IT5 bis IT6, Rundheitsabweichungen von 0,2µm, Zylindrizitätsabweichungen von weniger als 0,1µm sowie eine gemittelte Rauhtiefe $R_{z}$ von 0,5µm erreicht. Dies ist also ein deutlich genaueres Verfahren, was bei der Betrachtung der zu erwartenden Abweichungen unbedinngt in Erwägung gezogen werden muss. 

Die theoretische Oberflächenrauheit einer drehend hergestellten Oberfläche lässt sich laut [Dietrich.2014] und [Paucksch.2008] mit der Formel 

\begin{equation*}
	R_{t}=\frac{f^{2}}{r_{\epsilon}}
\end{equation*}

berechnen. In dieser Formel beschreibt $r_{\epsilon}$ den Eckenradius und $f$ den Vorschub des Werkzeugs in axialer Richtung. Die Vorschubgeschwindigkeit hat demnach einen quadratischen Einfluss auf die theoretische Oberflächenrauheit und gibt diese maßgeblich vor. Mit zunehmender Nutzungsdauer des Werkzeuges steigt dessen Verschleiß wobei sich somit der Betrag des Eckenradius stets ändert. Dies resultiert laut oben genannter Formel in einer Veränderung der Oberflächengüte des bearbeiteten Geometrieelements. 

Drehende Verfahren weisen eine starke gerichtete Bearbeitungsrichtung auf, weshalb durchweg gerichtete, rillige Oberflächen entstehen. Diese gerichteten Oberflächen weisen in der Regel quer zur Schnittrichtung größere Rauheiten als in Schnittrichtung auf.[Denkena]
Diese Rillen sind stark geprägt von der Form und dem Verschleißzustand des Drehmeißels sowie der Vorschubbewegung. [Paucksch.2008]
Weiterhin ist die Oberflächenrauheit geprägt von der Zerspanbarkeit des Werkstoffes. Eine gute Zerspanbarkeit ermöglicht geringe Fehler in der Form und Oberflächengüte des Bauteils. Werkstoffe mit einer homogenen Gefügeausbildung erreichen eine höhere Arbeitsgüte einfacher als Werkstücke mit Fehlstellen (z.B. Lunker oder Porositäten).

Wie alle Verfahren erzeugt auch das Drehen erfahrungsgemäß typische Fehler am Werkstück, die konkreten Ursachen zugeordnet werden können. Dies ermöglicht, den Rückschluss vom Fehler auf die Ursache und ermöglicht eine Anpassung der Prozessparameter um eine höhere Arbeitgenauigkeit zu steuern. 

Ungenügende Maßgenauigkeit lässt sich zum Beispiel über einen zu hohen Verschleiß des Werkzeuges erklären. Mit übermäßigem Verschleiß wird der Schneidkantenversatz des Drehmeißels zu groß, was eine kontrollierte Einstellung der Oberflächenrauheit erschwert. Eine weitere Ursache für zu hohe Maßabweichungen kann weiterhin in einer Durchbiegung des eingespannten Werkstückes liegen. Dies resultiert aus zu hohen Passivkräften bei der Zerspanung (zu hohe Schnittiefe oder Werkzeugvorschub), fehlender Abstützung schlanker Werkstücke oder einem zu hohen Einstellwinkel $\kappa$ des Werkzeuges. [Schönherr]

Ein weiterer Fehler, der sich bei drehend hergestellten Bauteilen finden lässt ist die Unrundheit des Profils außerhalb der vorgegebenen Toleranzen. Ein Grund dafür liegt in der übermäßigen Durchbiegung des Bauteils aus den oben genannten Gründen. Außerdem wirken sich eine ungenaue, außermittige Zentrierung, Längsführungen oder Haupspindellagerungen mit zu viel Spiel und elastische Verformungen des Werkstückes durch zu hohe Spannkräfte negativ auf die Rundheit des Bauteils aus.[Dietrich.2014], [Schonherr.2002] 

Bei Drehteilen lassen sich weiterhin wellige Oberflächen feststellen. Die ergibt sich aus Schwingungen der Drehmaschine und damit auch des Werkzeuges im Eingriff durch zu hohes Spiel in den Führungen der Maschine. Außerdem führen eine zu falsche Werkzeugeinspannung und zu hohe Schnittleistungen zum Einbringen von zusätzlichen Schwingungen in die Werkzeugmaschine, was die Bildung welliger Oberflächen begünstigt. [Dietrich.2014] 

Rattermarken am Werkstück entstehen aufgrund eines instabilen Werkzeugs (Auskraglänge zu groß, Schaftquerschnitt zu gering oder instabile Einspannung), zu hohen Passivkräften bei der Bearbeitung, einem zu hohen Freiflächenverschleiß oder einer ungenügenden Schneidkantenschärfe des Werkzeugs, beispielsweise wegen einer Beschichtung dessen. [Schonherr.2002]

Eine konische Form des angestrebten zylindrischen Bauteils entsteht laut [Dietrich] aufgrund einer nicht fluchtenden Anordnung von Drehachse der Maschine und der Achse der Spitze des Reitstocks. 

Kratzer auf der Oberfläche eines Drehteils lassen sich auf Oxidationsverschleiß der Nebenschneide des Werkzeugs oder auf den Kontakt mit Spänen zurückführen, welche die fertig gedrehte Oberfläche beschädigen. [Schonherr.2002] 

\subsubsection {Fertigungsabweichungen beim Bohren}

Beim Fertigungsverfahren Bohren sind verschiedene Maßtoleranzen und Rautiefen zu erreichen. Auch beim Bohren sind die Werte stark von dem Bohrverfahren und den Prozessparametern abhängig.
\prettyref{tab:bohrungsqualitaet} stellt den Sachverhalt anschaulich dar.

\begin{table}[h]	
	
	\begin{tabularx}{\columnwidth}{|X|c|c|l|}	
		
		
		\hline
		\textbf{Verfahren}&\textbf{Maßtoleranz}&\textbf{Rautiefe $R_{t}$ in µm}&\textbf{Oberflächenqualität}\\
		\hline
		Bohren ins Volle&IT12&80&Schruppen\\
		\hline
		Aufbohren mit Wendelsenkern&IT11&20&Schlichten\\
		\hline
		Senken mit Flach- und Form\-senkern&IT9&12&Schlichten\\
		\hline
		Reiben&IT7&8&Feinschlichten\\
		\hline
		Ausdrehen mit Ausdrehmeißel oder mehrschneidigem Bohrkopf&IT7&8&Feinschlichten\\
		\hline
		Ausdrehen mit Hartmetallschneiden und sehr kleinem Spanungsquerschnitt&IT7&4&Feinschlichten\\
		\hline	
		
	\end{tabularx}
	
	\caption{Maßtoleranzen und Rautiefen verschiedener Bohrverfahren (Quelle: Todo)}
	\label{tab:bohrungsqualitaet}

\end{table}

Des Weiteren beschreibt [Schoenherr] die Abhängigkeit der erreichbaren Genauigkeiten vom Material des Bohrers. So sind mit HSS-Spiralbohrern Toleranzen von IT12, mit Vollhartmetall-Spiralbohrern Toleranzen von IT9 und mit geradgenutetem Vollhartmetallbohrer Toleranzen von IT7 herstellbar. 

Generell ist aber zu sagen, dass ein Bohren ins Volle mit einem Wendelbohrer stets eine Schruppbearbeitung darstellt. Das heißt es ist keine hochgenaue Fertigung möglich. Viel bessere Genauigkeiten lassen sich mit Reiben, Senken, Feinbohren und Ausdrehen erzielen. [Schonherr.2002]

Die Formgenauigkeit einer Bohrung wird hauptsächlich durch deren Rundheit und Geradheit beschrieben. Mangelnde Formgenauigkeit kann verschiedene Ursachen haben. Dazu zählen eine nicht ausreichen hohe Steifigkeit von Werkzeugmaschine, Werkzeug und der Werkzeugaufnahme. Weiterhin wirken sich eine schlechte Zentrierung beim Anschnitt, ungleichmäßiges Schneiden aller Hauptschneiden des Bohrers und zu hohe Belastungen (Vorschub- und Schnittkräfte) negativ auf die Formhaltigkeit der Bohrung aus. Positiv können sich hingegen die Verwendung geradgenuteter oder rechtsgedrallter Bohrer und eine gute Spanbildung (Spanformung, Spanbruch, Spanabfuhr) niederschlagen. 

Von Werkzeugen verursachte Asymmetrien unterscheiden sich deutlich von werkstückbedingten Asymmetrien. Dadurch entstehen typische Bohrfehler.

Durch ungünstige Werkzeuge zu begründende Fehler bohrend hergestellter Flächen sind beispielsweise Überweite und Konizität des Geometrielements. Dabei sind nach [Paucksch] vor allem durch Anschlifffehler erzeugte asymmetrische Werkzeuge ursächlich. Diese sind charakterisiert durch Hauptschneidenlängenunterschiede, Einstellwinkeldifferenzen, Spitzenlängenabweichungen und Außermittigkeit der Werkzeugquerschneide.
Überweite kommt dabei am häufigsten vor. [Winkler] nennt neben dem ungünstigen Anschliff des Bohrers außerdem den Einsatz falscher Bohrerdurchmesser sowie den übermäßig langen und damit verschleißreichen Einsatz von Bohrern als weitere Gründe für eine Überweite der Bohrung. 
Weiterhin kann es bei stumpfen Bohrern zur Ausbildung einer Wulst an der Oberseite bei gleichzeitiger Gratbildung an der Unterseite kommen. Ebenfalls durch einen stumpfen Bohrer werden sehr raue Bohrungswandungen hervorgerufen. 
Hat das Werkzeug eine schlechte Führung, keine Zentrierspitze oder einen zu geringen Spitzenwinkel, so können in dünnwandigen Werkstücken unrunde Bohrungen entstehen. [Dietrich.2014]

Auch beim Bohren treten Abweichungen auf, die im Werkstück begründet sind. Ursachen dafür liegen nach [Paucksch] unter anderem in schrägen oder unebenen Flächen des Rohteils auf denen die Bohrung platziert werden soll. Zusätzlich dazu sind schräge Vorbohrungen, dünne Restwandstärken, Hohlräume, Querbohrungen sowie Lunker und Einschlüsse ursächlich dafür, dass es zu einer asymmetrischen Führung des Bohrers kommt. Daraus resultieren einseitige Kräfte auf das Werkzeug, was zu einem Ausweichen des Bohrers und dessen elastischer Verbiegung führt. Im Ergebnis verläuft der Bohrer und es kommt zur Ausbildung von Mittenabweichungen, schrägen Bohrungsachsen oder einer Unrundheit des Geometrieelements.
Weiterhin sorgen falsch platzierte Zentrierungen oder Vorbohrungen zu einer inkorrekten Lage der Bohrung.   

\subsubsection {Fertigungsabweichungen beim Fräsen}

Auch beim Fertigungsverfahren des Fräsens hängt die erreichbare Maßgenauigkeit und Oberflächengüte, wie bei allen anderen Fertigungsprozessen auch, stark vom konkret verwendeten Verfahren ab. \prettyref{tab:fraesqualität} zeigt diese Abhängigkeit. 

 
 \begin{table}[h]	
 	
 	\begin{tabularx}{\columnwidth}{|X|c|c|l|}	
 		
 		
 		\hline
 		\textbf{Verfahren}&\textbf{Maßtoleranz}&\textbf{Rautiefe $R_{t}$ in µm}\\
 		\hline
 		Walzenfräsen&IT8&30\\
 		\hline
 		Stirnfräsen&IT6&10\\
 		\hline
 		Formfräsen&IT7&20-30\\
 		\hline
 		
 	\end{tabularx}
 	
 	\caption{Maßtoleranzen und Rautiefen verschiedener Bohrverfahren (Quelle: Todo)}
 	\label{tab:fraesqualität}
 	
 \end{table}

Es ist, wie oben schon erwähnt, zu erkennen, dass deutlich maßhaltigere Werkstücke als beim klassischen Bohren mit Wendelbohrern herstellbar sind. 

Eine ungenügende Oberflächengüte beim Fräsen kann auf verschiedene Prozessparameter zurückzuführen sein. So deutet eine zu hohe Rautiefe auf zu geringe Schnittgeschwindigkeiten, einen zu großen Vorschub je Schneide, eine ungenügende Werkstückspannung zu große Schnittkräfte oder ein ratterndes Fräserwerkzeug infolge der Einbringung von Schwingungen aus der Maschine. [Dietrich.2014]
[Winkler1990] ergänzt, dass eine zu große Schneidkantenfase oder eine ungenügende Maschinenstabilität zu nicht ausreichender Oberflächenqualität führen können. 

Ein weitere typisches Fehlerbild, was beim Fräsen auftreten kann ist das Bilden von Kantenausbüchen. Dabei wurde der Vorschub pro Zahn zu groß gewählt oder eine zu große Schnittiefe eingestellt. Weiterhin ist ein zu großer Einstellwinkel und eine zu große Schneidkantenfase maßgebend für diesen Fehler.[Winkler.1990]

Zeigt die gefräste Oberfläche Vertiefungen in gleichen Abständen, dann ist nach [Dietrich.2014] ein schlagender Fräser die Ursache dafür. 

\subsubsection{Fertigungsabweichungen beim Räumen}

Beim Räumen sind nach [Dietrich.2014] und [Dedenke.2011] mit Sicherheit Maßgenauigkeiten von IT7 bis IT8 erreichbar. Mit einem erhöhten Aufwand lassen sich sogar Toleranzen von IT6 erzielen.

Die erreichbare Oberflächengüte beim Räumen wird maßgeblich vom letzten Schlichtzahn bestimmt.
Normalerweise erreicht man Oberflächentoleranzen $R_{z}$ von 6,3 bis 25µm. Mit besonders hohem Aufwand sind auch Werte von $R_{z} = 1mm$ realisierbar. Die Erzielung präziser Oberflächen ist sowohl bei Baustählen, als auch bei gut räumbaren Automatenstählen und Gusswerkstoffen möglich. Auch bei normalgeglühten Einsatz- und Vergütungsstählen mit einer gleichmäßigen Ferrit-Perlit-Verteilung sind diese Werte umsetzbar. [Dietrich.2014], [Dedenka.2011]       

Ist die Vorbohrung, durch welche die Räumnadel gezogen wird, zu groß, kann es sowohl zum Verlaufen des Werkzeugs, als auch zu großen Lageabweichungen des Geometrieelements kommen. [Denkena.2011]
Übermäßige Lageabweichungen sind weiterhin typisch für während des Vorganges schwimmend gelagerte Werkstücke und sehr schlanke Teile, die eine geringe Quersteigfigkeit aufweisen. 

Ist die Auflage des Werkstücks nicht rechtwinklig zur Bohrung, kann es zu Rattermarken mit großen Abständen kommen. 
Ursache für Quetscherscheinungen an der Auslaufseite des Werkstücks liegen in weichen Stellen innerhalb des Werkstoffes begründet. [Dietrich.2014]    

\subsubsection{Fertigungsabweichungen beim Schleifen}

Kommt es beim Schleifen zu Maßungenauigkeiten liegt dies an einer zu großen Schleifzugabe, ungenügender Kühlung, einer nicht ausgewuchteten oder nicht ausreichend abgerichteten Schleifscheibe. 

Formfehler treten dann auf, wenn das Werkstück vor dem Schleifen nicht optimal ausgerichtet oder das Werkzeug ungünstig gewählt wurde. Dazu zählen ein zu grobes Korn, eine zu niedrige Härte und ein unzureichend dichtes Gefüge der Schleifscheibe. Außerdem kann auch die Abrichteinheit fehlerhaft sein, was ebenfalls zu Formfehlern am Bauteil führt. [Winkler.1990] 

Weist das Bauteil eine zu große Rautiefe auf, dann liegt das nach [Winkler.1990] in einem der folgenden Ursachen begründet. Sowohl eine zu niedgrige Schnittgeschwindigkeit, eine zu hohe Einstechgeschwindigkeit der Scheibe, eine zu geringe Ausfeuerzeit der Scheibe, eine zu hohe Korngröße, ein zu geringer Schmierölanteil, eine zu hohe Abrichtgeschwindigkeit oder eine zu hohe Breite des Abrichtwerkzeugs wirken sich negativ auf die erreichbare Oberflächentoleranz aus. 

Ein typischer Fehler, der beim schleifenden Bearbeiten auftritt, sind die sogenannten Schleifriefen. Ursächlich dafür sind ein zu gober Schleifkörper und eine zu geringe Ausfeuerzeit der Schleifscheibe.[Dietrich.2014]      
Ergänzend dazu nennt [Winkler.1990] ein zu hohes Aufmaß des Werkstücks, einen zu niedrigen Fettanteil des Schmiermittels, unzureichende Kühlung und eine zu harte Schleifscheibe. Weiterhin sind eine zu hohe Einstechgeschwindigkeit, eine zu hohe Schnittgeschwindigkeit, ein zu großer Abrichtbetrag, eine zu niedrige Abrichtgeschwindigkeit und ein zu breites Abrichtwerkzeug für dieses Fehlerbild verantwortlich.

Weiterhin sind Rattermarken typisch für geschliffene Oberflächen. Eine zu harte Schleifschiebe und in den Prozess eingebrachte Schwingungen verursachen diese. Diese Schwingungen können zum Beispiel durch Führungsbahnen der Maschine mit zu viel Spiel oder eine nicht ausreichend ausgewuchtete Schleifscheibe entstehen und durch eine zu harte Schleifscheibe ungedämpft in den Zerspanungsvorgang eingebracht werden. Eine zu hohe Einstechgeschwindigkeit, ein zu großer Vorschub und eine zu hohe Schnittgeschwindigkeit begünstigen ebenso das Auftreten von Rattermarken auf der geschliffenen Oberfläche.     





 

      





 

   
     



      








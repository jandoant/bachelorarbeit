\section{Aufgabenstellung}

Alle realen Fertigungsverfahren erzeugen gewisse charakteristische geometrische Abweichungen von der Nenngestalt, die zum Beispiel einem zeitlichen Trend oder einer zufälligen Auftretenswahrscheinlichkeit folgen. Ursache für diese Abweichungen sind dem jeweiligen Fertigungsverfahren innewohnende, teils unvermeidliche Erscheinungen. Zu diesen gehören zum Beispiel der Verschleiß des Werkzeugs oder äußere Schwingungen. 

Die beherrschte Vorhersage solcher Abweichungen ist bedeutend zur Absicherung der Produktqualität über die gesamte Produktionsstückzahl. Die Kenntnis über die sich wahrscheinlich einstellenden Ist-Abweichungen der Nenngestalt ist wichtig für verschiedene Prozesse der Fertigungsmesstechnik, wie zum Beispiel zur Definition der Messstrategie (Messpunktmuster, Messpunktanzahl, etc.). 

Ziel der Abschlussarbeit ist einerseits die umfassende Zusammenstellung und Kategorisierung charakteristischer geometrischer Fertigungsabweichungen. Darauf aufbauend ist ein geeignetes Werkzeug zu generieren, um die zu erwartenden Abweichungen durch eine erzeugte Punktewolke darzustellen.
Das geschieht, indem die vorhandene Soll-Geometrie durch einen geeigneten Algorithmus mit den erarbeiteten typischen Fertigungsfehlern beaufschlagt werden können. Dafür wird die Umsetzung mittels MATLAB oder einer geeigneten CAD-Software vorgeschlagen. 

\paragraph{Teilaufgaben}

\begin{itemize}
	\item Recherche und Katalogisierung von charakteristischen geometrischen Abweichungen, die aus realen Fertigungsverfahren hervorgehen können
	\item Erarbeitung eines Werkzeugs zur Erzeugung von erwartbaren Messpunktwolken durch die Beaufschlagung der Soll-Geometrie mit geometrischen Fertigungsabweichungen, zum Beispiel mittels MATLAB oder CAD
	\item Zusammenstellung der Erkenntnisse und Erarbeitung eines Ausblicks  
\end{itemize}
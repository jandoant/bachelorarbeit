\section{Fazit und Ausblick}

Mit den Überlegungen in dieser Arbeit und dem entwickelten Programm wurde eine Grundlage für ein Software-System geschaffen, welches perspektivisch, erwartbare Fertigungsabweichungen bei der Herstellung von Bauteilen simulieren kann.

Es wurde untersucht, welche typischen Abweichungen bei der Fertigung von Werkstücken mit bestimmten Verfahren auftreten. Dabei ist festzustellen, dass qualitativ durchaus charakteristische Fehlerbilder für verschiedene Fertigungsprozesse zu erkennen sind. Allerdings lassen die Untersuchungen nur begrenzt qualitative Aussagen bezüglich der Fehler zu. Das erschwert eine Konzeption der Software mit Hinsicht auf die Zielstellung, vom angewandten Fertigungsverfahren direkt auf die zu erwartenden Fehler zu schließen. Es ist aber realistisch, die Art der möglichen Fehler einer Oberfläche nach dem Fertigungsverfahren zu klassifizieren und dem Nutzer bei Auswahl der jeweiligen Körperfläche anzubieten. Die konkreten Werte müssen vom Nutzer auf Grundlage der spezifischen Herstellungseinflüsse selbst ermittelt und getroffen werden. Die zu erwartende Wirkliche Oberfläche des Bauteils ist quantitativ nicht generalisiert vorhersagbar. Mit Kenntnis der Arbeitsweise der verwendeten Werkzeugmaschine, der Produktionsumgebung und den Prozessbedingungen sowie dem Material und dem Verschleißzustandes des Werkzeuges lassen sich durch Vergleich mit, unter ähnlichen Bedingungen bereits hergestellten Bauteilen, Vorhersagen treffen.
Diese für jeden Fertigungsprozess zu erlangenden Erkenntnisse müssen in die Software einfließen, sodass zu erwartende Fehler kundenspezifisch simuliert werden können. 

Das entwickelte Programm ist zum derzeitigen Stand in der Lage eine STEP-Datei des zu untersuchenden Bauteils einzulesen und in einzelnen Geometrieelemente zu zerlegen. Für jedes einzelne Geometrieelement lassen sich kontinuierliche Deformationsfunktionen anwenden, welche die ideale Formgestalt des Bauteils ändern. Als Ergebnis werden die räumlichen Koordinaten einer Punktewolke der veränderten Geometrie, in einer Text-Datei ausgegeben. 

Es ist zu betonen, dass die vorliegende Software als Grundlage zur Umsetzung einer Simulationssoftware für Fertigungsabweichungen eines vorgegebene Bauteils zu betrachten ist. Es handelt sich keineswegs um ein bereits praktisch verwendbares Programm.   
Zum derzeitigen Stand sind nur Bauteile, die entweder planare  oder zylindrische Flächen als Geometrieelemente aufweisen, verwendbar. Um andere Geometrien, wie beispielsweise Freiformflächen oder gezogene Profile verwenden zu können, müsste auf Grundlage der bereits umgesetzten STEP-Entitäten eine Ergänzung der Implementierung anderer Körperflächentypen geschehen. 
Das entwickelte Programm bietet die Möglichkeit eine ideale Geometrie mit gewünschten Verformungen zu beaufschlagen. Allerdings muss der Katalog an möglichen Deformationsfunktionen erweitert werden. Die bisher verfügbaren Funktionen sollen die angedachte Funktionsweise des Programms verdeutlichen, sind aber noch nicht an die eigentlichen Fertigungsfehler angepasst und optimiert. Um eine tatsächliche Simulationsfunktion der Software umzusetzen müssen solche Funktionen gefunden werden, die eine Anpassung der Parameter durch den Nutzer erlauben und realistische Fehler abbilden können. 

In der Weiterentwicklung des Programms gibt es sehr viel Potential. Neben der angesprochenen Erweiterung des Spektrums an Geometrieelementen und möglichen Deformationen ist auch die Benutzerfreundlichkeit ein wichtiger Aspekt. So ist eine Visualisierung der Geometrie unbedingt notwendig, um die Auswahl der zu verändernden Geometrieelemente für den Nutzer intuitiv zu gestalten. Derzeit ist eine Auswahl einer Körperfläche nur über deren ID möglich, dies stellt aber eine sehr umständliche und wenig anwenderfreundliche Art der Bedienung dar. Des weiteren muss ein konkreter Programmablauf, ähnlich dem dargestellten Beispielprogrammablauf, entwickelt werden. Die Bausteine und Herangehensweise,um dies umzusetzen, sind grundsätzlich durch die Programmierung in dieser Arbeit gegeben.    

Durch die Verwendung kontinuierlicher Funktionen können hauptsächlich Gestaltabweichungen nach DIN 4760 \cite{DIN.4760} umgesetzt werden. Es ist aber perspektivisch auch denkbar Oberflächenunvollkommenheiten nach DIN EN ISO 8785 \cite{DIN.8785} zu simulieren. Mit dem aus der Computergrafik bekannten Verfahren des Displacement Mappings, wie beispielsweise in \cite{Kalos.2008} beschrieben, lassen sich über diskret, zum Beispiel in Form von Graustufenbildern, gegebene Verformungsfelder auf Körperflächen projizieren. Dabei ist in jedem Pixel des Graustufenbildes ein Helligkeitswert gespeichert, der ein Maß für die Verschiebung des Punktes an der Stelle des Pixels darstellt. So lassen sich auch singuläre Strukturen, wie Kratzer oder Ausbrüche auf der Körperoberfläche darstellen. 

Der derzeitig vorhandene Prototyp des Programms muss und kann also noch in vielen Aspekten weiterentwickelt werden um eine aussagekräftige Simulation typischer Fertiungsabweichungen zu erlauben.    

      



          